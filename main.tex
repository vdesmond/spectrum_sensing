\documentclass[a4paper]{article}
\usepackage[document]{ragged2e}
%\usepackage{simplemargins}

%\usepackage[square]{natbib}
\usepackage{amsmath}
\usepackage{amsfonts}
\usepackage{amssymb}
\usepackage{graphicx}
\usepackage{tabularx}
\usepackage[export]{adjustbox}
\usepackage[numbered,framed]{matlab-prettifier}
\usepackage{minted}
\usepackage{svg}
\usepackage{graphicx}
\usepackage{fancyhdr}
\pagestyle{fancy}
\usepackage{xcolor}
\usepackage{lipsum}
\setlength\headheight{26pt} %% just to make warning go away. Adjust the value after looking into the warning.
\lfoot{{\small\textsf{\thepage}}}
\rfoot{\small\textsf{\today}}
\cfoot{}
\lhead{\includegraphics[width=30pt]{ssn}}
\begin{document}
% \pagenumbering{gobble}

\Large
 \begin{center}
UEC1728 - Cognitive Radio\\
\textbf{Assignment}

\hspace{10pt}

% Author names and affiliations
\large
Vignesh Mohan\\
\smallskip
\small
183002181\\
\small
Electronics and Communication Department\\
\small
Sri Sivasubramiya Nadar College of Engineering, 603110\\
\end{center}

\hspace{10pt}
\normalsize
\numberwithin{figure}{section}

%%%%%%%%%%%%%%%%%%%%%%%%%%%%%%%%%%%%%%%%%%%%%%%%%%%%%%%%%%%%%%%%%%%%%%%%%%%%%%%%%%%%%%%%%%%%
\section{Energy Detection}
\textbf{\large{Aim}}:\\[10pt]
To detect the unknown PU (Primary user) signal based on energy of the
received signal. The Energy detector is optimal to detect the unknown signal
if the noise power is known. In the energy detection, CR users sense the
presence/absence of the PUs based on the energy of the received signals. As
above, the measured signal r (t) is squared and integrated over the observation
interval T. Finally, the output of the integrator’s compared with a threshold k
to decide if a PU is present. Since the energy detection depends only on the
SNR of the received signal, its performance is susceptible to uncertainty in
noise power. If the noise power is uncertain, the energy detector will not be
able to detect the signal reliably as the SNR is less than a certain threshold,
called an SNR wall. In addition, the energy detector can only determine
the presence of the signal but cannot differentiate signal types. Thereby, the
energy detector often results in false detection triggered by the unintended CR
signals.\\[10pt]
\textbf{i) Code for performance of detector (pd) under various values of probability
of false alarm (pfa) for SNR = 6dB}\\
\definecolor{bg}{rgb}{0.95,0.95,0.95}
\begin{minted}[linenos=true,bgcolor=bg]{matlab}
% energy_detection_pfa.m
clc;
close all;
m = 10;
N = 2 * m;
pf = logspace(-5,0,20);
snr_avgdB = 4;
snr_avg = power(10, snr_avgdB / 10);
for i = 1:length(pf)
    over_num = 0;
    th(i) = gammaincinv(1 - pf(i), 2 * m);
    for kk = 1:1000
        t = 1:N;
        x = sin(pi * t);
        noise = randn(1, N);
        pn = mean(noise.^2);
        amp = sqrt(noise.^2 * snr_avg);
        x = amp .* x ./ abs(x);
        SNRdB_Sample = 10 * log10(x.^2 ./ (noise.^2));
        signal = x(1:N);
        ps = mean(abs(signal).^2);
        Rev_sig = signal + noise;
        accum_power(i) = sum(abs(Rev_sig)) / pn;
        if accum_power(i) > th(i)
            over_num = over_num + 1;
        end
    end
    pd_sim(i) = over_num / kk;
end
figure;
s = semilogy(pf, pd_sim, 'r-*');
set(s, 'linewidth', 1);
title('Nonfluctuating Coherent ROC');
grid on;
xlabel('Probability of Falsed Alarm (pfa)');
ylabel('Probability of Detection (pd)');
legend(['SNR=' num2str(snr_avgdB) 'dB']);
\end{minted}
\textbf{ii) Code for performance of detector under various values of signal to noise ratio (SNR)}\\
\begin{minted}[linenos=true,bgcolor=bg]{matlab}
% energy_detection_snr.m
clc;
close all;
m = 10;
N = 2 * m;
pd_sim = [];
pf = 0.5;
for snr_avgdB = 0:1:30
    snr_avg = power(10, snr_avgdB / 10);
    for i = 1:length(pf)
        over_num = 0;
        th(i) = gammaincinv(1 - pf(i), 2 * m);
        for kk = 1:1000
            t = 1:N;
            x = sin(pi * t);
            noise = randn(1, N);
            pn = mean(noise.^2);
            amp = sqrt(noise.^2 * snr_avg);
            x = amp .* x ./ abs(x);
            SNRdB_Sample = 10 * log10(x.^2 ./ (noise.^2));
            signal = x(1:N);
            ps = mean(abs(signal).^2);
            Rev_sig = signal + noise;
            accum_power(i) = sum(abs(Rev_sig)) / pn;
            if accum_power(i) > th(i)
                over_num = over_num + 1;
            end
        end
        pd_sim = [pd_sim over_num / kk];
    end
end
figure;
SNR = 0:1:30;
s = plot(SNR, pd_sim, 'r-*');
set(s, 'linewidth', 1);
title('Energy detection method');
grid on;
xlabel('Signal-to-noise ratio (dB)');
ylabel('Probability of Detection (pd)');
legend(['PFA=' num2str(pf)]);
\end{minted}
\textbf{\large{Results and Inference}}:\\[10pt]
\begin{figure}[h!bt]
  \centering
  \includesvg[width=1\linewidth]{ed_pfa}
  \caption{Pd Vs Pfa for Energy Detection (SNR = 6dB)}
  \label{fig:1.1}
\end{figure}
\begin{figure}[h!bt]
  \centering
  \includesvg[width=1\linewidth]{ed_snr}
  \caption{Pd Vs SNR for Energy Detection (Pfa = 0.5)}
  \label{fig:1.2}
\end{figure}
The observation from Figure \ref{fig:1.1} is the presence of a tradeoff between Pd and Pfa values. For Pfa values from 0.09 to 0.8 the detection probability is optimum. After that the detection probability approaches 1 for SNR = 4 db.\\[4pt]
The observation from Figure \ref{fig:1.2} is the presence of a linear increase in the probability of detection for increases in SNR value. For low values of SNR the detection probability is almost 0. Above 8 db the detection probability approaches 1.\\
%%%%%%%%%%%%%%%%%%%%%%%%%%%%%%%%%%%%%%%%%%%%%%%%%%%%%%%%%%%%%%%%%%%%%%%%%%%%%%%%%%%%%%%%%%%%
\section{Matched Filter Detection}
\textbf{\large{Aim}}:\\[10pt]
To detect the unknown PU (Primary user) signal based on Prior knowledge
(bandwidth, operating frequency, modulation type etc) and to maximize the
SNR. When the user has information on the primary user signal, the optimal
detector in stationary Gaussian noise happens to be the matched filter as it
maximizes the reclosed signal-to-noise ratio (SNR). The palpable advantage
of matched filter lies in its requirement of smaller time for achieving high
processing gain for the reason of coherence and of a prior knowledge of the
basic user signal like the modulation type and order, the shape of the pulse
and the packet format. The performance of the matched filter will be poor if
this information has no accuracy. Most wireless networks system have pilot,
preambles, synchronization word or spreading codes and hence can be used
for coherent detection.\\[10pt]
\textbf{i) Code for performance of detector (pd) under various values of probability
of false alarm (pfa) for SNR = 6dB}\\
\begin{minted}[linenos=true,bgcolor=bg]{matlab}
% matched_filter_detection_pfa.m
clc;
close all;
rstream = RandStream.create('mt19937ar', 'seed', 2009);
Ntrial = 1e5; % number of Monte-Carlo trials
spower = 1; % signal power is 1
pd_graph = [];
for snrdb = 6 % SNR in dB
    snr = db2pow(snrdb); % SNR in linear scale
    npower = spower / snr; % noise power
    namp = sqrt(npower / 2); % noise amplitude in each channel
    s = ones(1, Ntrial); % signal
    n = namp * (randn(rstream, 1, Ntrial) + 1i...
        * randn(rstream, 1, Ntrial)); % noise
    x = s + n;
    mf = 1;
    y = mf' * x;
    z = real(y);
    Pfa = 1e-3;
    snrthreshold = npwgnthresh(Pfa, 1, 'coherent');
    mfgain = mf' * mf;
    % To match the equation in the text above
    % npower - N
    % mfgain - M
    % snrthreshold - SNR
    threshold = sqrt(npower * mfgain * snrthreshold);
    Pd = sum(z > threshold) / Ntrial;
    x = n;
    y = mf' * x;
    z = real(y);
    Pfa = sum(z > threshold) / Ntrial;
    [Y, X] = rocsnr(snrdb, 'SignalType', ...
        'Nonfluctuatingcoherent', 'MinPfa', 1e-4);
    pd_graph = [pd_graph; Y'];
end
figure;
s=semilogx(X,pd_graph(1,:),'r-*');
set(s, 'linewidth', 1);
title('Nonfluctuating Coherent ROC - Matched Filter');
grid on;
xlabel('probability of falsed alarm(pfa)');
ylabel('probability of detection(pd)');
snr_avgdB = 6;
legend(['SNR=' num2str(snr_avgdB(1)) 'dB'])
\end{minted}
\textbf{ii) Code for performance of detector under various values of signal to noise ratio (SNR)}\\
\begin{minted}[linenos=true,bgcolor=bg]{matlab}
% matched_filter_detection_snr.m
clc;
close all;
rstream = RandStream.create('mt19937ar', 'seed', 2009);
spower = 1; % signal power is 1
Ntrial = 1e5; % number of Monte-Carlo trials
s = ones(1, Ntrial); % signal
mf = 1;
Pfa = 0.5;
Pd = [];
for snrdb = 0:1:30 % SNR in dB
    snr = db2pow(snrdb); % SNR in linear scale
    npower = spower / snr; % noise power
    namp = sqrt(npower / 2); % noise amplitude in each channel
    n = namp * (randn(rstream, 1, Ntrial) + 1i ...
        * randn(rstream, 1, Ntrial)); % noise
    x = s + n;
    y = mf' * x; % apply the matched filter
    z = real(y);
    snrthreshold = npwgnthresh(Pfa, 1, 'coherent');
    mfgain = mf' * mf;
    threshold = sqrt(npower * mfgain * snrthreshold);
    Pd = [Pd sum(z > threshold) / Ntrial];
end
figure;
SNR = 0:1:30;
s = plot(SNR, Pd, 'r-*');
set(s, 'linewidth', 1);
title('Matched Filter');
grid on;
xlabel('Signal-to-noise ratio (dB)');
ylabel('probability of detection(pd)');
legend(['PFA=' num2str(Pfa)]);
\end{minted}
\textbf{\large{Results and Inference}}:\\[10pt]
\begin{figure}[h!bt]
  \centering
  \includesvg[width=1\linewidth]{mfd_pfa}
  \caption{Pd Vs Pfa for Matched Filter Detection (SNR = 6dB)}
  \label{fig:2.1}
\end{figure}
\begin{figure}[h!bt]
  \centering
  \includesvg[width=1\linewidth]{mfd_snr}
  \caption{Pd Vs SNR for Matched Filter Detection (Pfa = 0.5)}
  \label{fig:2.2}
\end{figure}
The observation from the Figure \ref{fig:2.1} is the presence of a trade off between
Pd and Pfa values. For SNR = 6dB, the detection probability approaches
1after Pfa = 0.5. The observation from the Figure \ref{fig:2.2} is the presence of a linear increase in the probability of detection for increasing SNR values. The detection probability
is high even for low values of SNR. Above 5 db the detection probability
reaches 1.
%%%%%%%%%%%%%%%%%%%%%%%%%%%%%%%%%%%%%%%%%%%%%%%%%%%%%%%%%%%%%%%%%%%%%%%%%%%%%%%%%%%%%%%%%%%%
\newpage
\section{Cyclostationary Feature Detection}
\textbf{\large{Aim}}:\\[10pt]
To detect the PU (Primary user) signal exploiting the cyclostationary features
(spreading code, cyclic prefixes etc). Cyclostationary feature detection is
an alternative detection method. Modulated signals are combined with sine
wave carriers, pulse trains, repeating spreading, hopping sequences or cyclic
prefixes which end up in built in periodicity. These modulated for the reason
that their mean and autocorrelation display periodicity feature. Analysis of
the spectral correlation function helps detection of such periodicity feature.
The spectral correlation function can distinguish between noise energy and
modulated signal energy. This arises from the feature of noise as a wide-sense
stationary signal and no correlation. On the other hand, modulated signals
have the feature of being cyclostationary with special correlation in view of
the embedded redundancy of signal periodicity. Therefore, it is possible for a
detector of cyclostationary feature top perform better than the energy detector
for discrimination against noise. This is because of its robustness to uncertainty
seen in noise power. But, there is the computational complexity entailing a
long duration for observation.\\[10pt]
\textbf{Code for Cyclostationary Feature Detection}\\
\begin{minted}[linenos=true,bgcolor=bg]{matlab}
% cyclostationary_feature_detection.m
clc;
close all;
r = randi([0 1], 100, 1);
stem(r, 'DisplayName', 'r');
hold on;
l = length(r);
r(l + 1) = 0;
n = 1;
L = 1000;
fs = 1000;
T = 1 / fs;
fc = 3 / T;
t1 = (0:L);
while n <= l
    t = (n - 1):.001:n;
    if r(n) == 1
        if r(n + 1) == r(n)
            y = (t <= n);
        else
            y = (t < n);
        end
    else
        if r(n + 1) == r(n)
            y = (t > n);
        else
            y = (t >= n);
        end
    end
    plot(t, y, 'r');
    title('NRZ encoding');
    grid on;
    xlabel('time');
    ylabel('amplitude');
    hold on;
    axis([0 100 -1.5 1.5]);
    n = n + 1;
end
b = (2 * 3.142 * fc * t1) + (3.142 * (1 - y));
s = sqrt(2) * cos(b);
figure;
plot(t1, s, 'r');
axis([0 1000 -1.5 1.5]);
title('BPSK - Transmitted');
xlabel('Time');
ylabel('Amplitude');
axis([0 1000 -1.5 1.5])
hold on;
noise = awgn(s, 20, 'measured');
snr_db = 10 * log10(s.^2 / (noise.^2));
snr_avg = power(10, snr_db / 10);
rs1 = (s + noise)/2;
vr = var(rs1);
figure;
plot(t1, rs1, 'r');
axis([0 100 -1.5 1.5]);
title('BPSK - Received');
xlabel('Time');
ylabel('Amplitude');
axis([0 1000 -1.5 1.5])
hold on;
nfft = 2^nextpow2(L);
f1 = fft(rs1, nfft) / L;
fs1 = fs / 2 * linspace(0, 1, nfft / 2 + 1);
figure;
plot(fs1, 2 * abs(f1(1:nfft / 2 + 1)), 'r');
title('FFT of received signal');
xlabel('Frequency (HZ)');
ylabel('|rs1|');
grid on;
hold on;
fc=1:3/T;
s1=cos(2*3.14*fc*T);
l1=length(s1);
sc=s1+sin(2*3.14*fc*T);
l2=length(sc);
cr1=xcorr2(rs1,s1);
figure;
plot(cr1, 'r');
title('Cross Correlation of Rx signal with cos');
xlabel('Frequency (Hz)');
ylabel('Amplitude');
axis([0 500 -5.5 5.5]);
hold on;
cr2=xcorr2(rs1,sc);
l2=length(cr2);
figure;
plot(cr2,'r');
title('Cross Correlation of Rx signal with with sin+cos');
xlabel('Frequency (Hz)');
ylabel('Amplitude');
axis([0 500 -5.5 5.5]);
hold on;
for i = 1:500
    if s1(i) > 0.5 || sc(i) > 0.5
        disp('PU PRESENT');
    end
end
L = 2:1:30;
snr1 = 2:1:30;
q1 = (2 * L + 1) * (0.5)^2;
Pf = exp(-q1);
del = (2 * snr1 + 1) * vr / (2 * L + 1);
q2 = sqrt(2 * snr1 / vr);
q3 = 0.5 / del;
Pd = marcumq(q1, q2);
figure;
P = plot(snr1, Pd, 'r-*');
set(P, 'linewidth', 1);
title('Pd Vs SNR');
xlabel('SNR');
ylabel('Probability of detection');
hold on;
grid on;
figure;
P2 = semilogx(Pf, Pd, 'r-*');
set(P2, 'linewidth', 1);
title('Pfa Vs Pd');
xlabel('Probability of false alarm');
ylabel('Probability of detection');
hold on;
grid on;
\end{minted}
\newpage
\textbf{\large{Results and Inference}}:\\[10pt]
\begin{figure}[h!bt]
  \centering
  \includesvg[width=1\linewidth]{cfd_nrz}
  \caption{NRZ Encoding of input data}
  \label{fig:3.1}
\end{figure}
\begin{figure}[h!bt]
  \centering
  \includesvg[width=1\linewidth]{cfd_bpsk_tx}
  \caption{BPSK - Transmitted signal}
  \label{fig:3.2}
\end{figure}
\begin{figure}[h!bt]
  \centering
  \includesvg[width=1\linewidth]{cfd_bpsk_rx}
  \caption{BPSK - Received signal with AWGN noise}
  \label{fig:3.3}
\end{figure}
\begin{figure}[h!bt]
  \centering
  \includesvg[width=1\linewidth]{cfd_fft}
  \caption{FFT of Received signal showing main frequency component}
  \label{fig:3.4}
\end{figure}
\begin{figure}[h!bt]
  \centering
  \includesvg[width=1\linewidth]{cfd_ccorr_cos}
  \caption{Cross Correlation of received signal with cosine}
  \label{fig:3.5}
\end{figure}
\begin{figure}[h!bt]
  \centering
  \includesvg[width=1\linewidth]{cfd_ccorr_sincos}
  \caption{Cross Correlation of received signal with sine + cosine}
  \label{fig:3.6}
\end{figure}
\begin{figure}[h!bt]
  \centering
  \includesvg[width=1\linewidth]{cfd_pfa}
  \caption{Pd Vs Pfa for Cyclostationary Feature Detection}
  \label{fig:3.7}
\end{figure}
\begin{figure}[h!bt]
  \centering
  \includesvg[width=1\linewidth]{cfd_snr}
  \caption{Pd Vs SNR for Cyclostationary Feature Detection}
  \label{fig:3.8}
\end{figure}
The observation from the Figures 7.5 and 7.6 is the presence of a linear increase
in the probability of detection for increasing SNR values. For SNR = 2 dB,
probability of detection is increased compared to energy and matched filter
detection. Finally, the detection probability reaches 1 at SNR = 10 db.\\[4pt]
The observation from the Figure 7.7 is the presence of a tradeoff between Pf
and Pd. For the lower values of Pf up to .001, probability of detection is 1and
detection of PU is less for more values of false alarm.\\
%%%%%%%%%%%%%%%%%%%%%%%%%%%%%%%%%%%%%%%%%%%%%%%%%%%%%%%%%%%%%%%%%%%%%%%%%%%%%%%%%%%%%%%%%%%%
\newpage
\section{Co-Operative Spectrum Sensing}
\textbf{\large{Aim}}:\\[10pt]
To enhance the sensing performance by exploiting the spatial diversity in
the observations of spatially located CR users. Theoretically, there is greater
accuracy seen in cooperative detection among unlicensed users with the
possibility of uncertainty in a single user’s detection getting minimized.
The features of multipath fading and shadowing effect undermine the performance of primary user detection methods. However, cooperative detection schemes permit mitigation of multipath fading and shadowing effects, thereby improving the detection probability in a heavily shadowed ambience.\\[10pt]
\textbf{Code for Co-Operative Spectrum Sensing}\\
\begin{minted}[linenos=true,bgcolor=bg]{matlab}
% cooperative_spectrum_sensing.m
clc;
close all;
u = 1000; % time bandwidth factor
N = 2 * u; % samples
a = 2; % path loss exponent
C = 2; % constant losses
Crs = 10; % Number of cognitive radio users
PdAnd = 0;
% ----------Pfa------------%
Pf = 0:0.005:1;
Pfa = Pf.^2;
% ---------signal-----%
t = 1:N;
s1 = cos(pi * t);
% s1power=var(s1);
% -------- SNR ----------%
% Snrdb=-15:1:15;
Snrdb = 15;
Snreal = power(10, Snrdb / 10); % Linear Snr
% while Snrdb<15
for i = 1:length(Pfa)
    lamda(i) = gammaincinv(1 - Pfa(i), u) * 2; % thershold
    % lamdadB=10*log10(lamda);
    % ---------Local spectrum sensing---------%
    for j = 1:Crs % for each node
        detect = 0;
        % d(j)=7+1.1*rand(); %random distanse
        d = 7:0.1:8;
        PL = C * (d(j)^-a); % path loss
        for sim = 1:10 
        % Monte Carlo Simulation for 100 noise realisation
            % -------------AWGN channel--------------------%
            noise = randn(1, N);
            % Noise production with zero mean and s^2 var
            noise_power = mean(noise.^2); % noise average power
            amp = sqrt(noise.^2 * Snreal);
            s1 = amp .* s1 ./ abs(s1);
            % SNRdB_Sample=10*log10(s1.^2./(noise.^2));
            Rec_signal = s1 + noise; % received signal
            localSNR(j) = mean(abs(s1).^2) * PL / noise_power;
            Pdth(j, i) = marcumq(sqrt(2 * localSNR(j))...
                , sqrt(lamda(i)), u); % Pd for j node
            % Computation of Test statistic for energy detection
            Sum = abs(Rec_signal) * PL;
            Test(j, sim) = sum(Sum);
            if Test(j, sim) > lamda(i)
                detect = detect + 1;
            end
        end % END Monte Carlo
        Pdsim(j) = detect / sim; 
        % Pd of simulation for the j-th CRuser
    end
    PdAND(i) = prod(Pdsim);
    PdOR(i) = 1 - prod(1 - Pdsim);
end
PdAND5 = (Pdth(5, :)).^5;
Pmd5 = 1 - PdAND5;
PdANDth = (Pdth(Crs, :)).^Crs;
PmdANDth = 1 - PdANDth; % Probability of miss detection
Pmdsim = 1 - PdAND;
figure(1);
semilogy(Pfa, Pmdsim, 'r-x', ...
    Pfa, PmdANDth, 'k-o', Pfa, Pmd5, 'g-*');
title('Complementary ROC Cooperative sensing - AND rule (AWGN)');
grid on;
ylim([0,1]);
xlabel('Probability of False alarm (Pfa)');
ylabel('Probability of Missed Detection (Pmd)');
legend('Simulation', 'Theory n=10', 'Theory n=5');
\end{minted}
\textbf{\large{Results and Inference}}:\\[10pt]
\begin{figure}[h!bt]
  \centering
  \includesvg[width=1\linewidth]{css}
  \caption{Complementary ROC of Cooperative sensing - AND rule under AWGN channel}
  \label{fig:4.1}
\end{figure}
The observation from the Figure \ref{fig:4.1} is that, observed that, for lower values of
Pfa probability of missed detection is 1 and as the value increases probability
of missed detection value gradually decreases. As the number of cognitive
radio users is increased from 5 to 10, the performance of missed detection is
increased but there is a trade-off between simulation and theory performance.
In results of simulation, Pmd value remains 1 up to 0.99 value of Pfa and later
approaches 1.
\end{document}